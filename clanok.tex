% Metódy inžinierskej práce

\documentclass[10pt,twoside,slovak,a4paper]{article}

\usepackage[slovak]{babel}
%\usepackage[T1]{fontenc}
\usepackage[IL2]{fontenc} % lepšia sadzba písmena Ľ než v T1
\usepackage[utf8]{inputenc}
\usepackage{graphicx}
\usepackage{url} % príkaz \url na formátovanie URL
\usepackage{hyperref} % odkazy v texte budú aktívne (pri niektorých triedach dokumentov spôsobuje posun textu)

\usepackage{cite}
%\usepackage{times}

\pagestyle{headings}

\title{Umelá inteligencia vo videohrách\thanks{Semestrálny projekt v predmete Metódy inžinierskej práce, ak. rok 2022/23, vedenie: Igor Stupavský}} % meno a priezvisko vyučujúceho na cvičeniach

\author{Dávid Pilný\\[2pt]
	{\small Slovenská technická univerzita v Bratislave}\\
	{\small Fakulta informatiky a informačných technológií}\\
	{\small \texttt{xpilnyd@stuba.sk}}
	}

\date{\small dátum} % upravte



\begin{document}

\maketitle

\begin{abstract}
Prvá iterácia umelej inteligencie vo videohrách sa objavila v roku 1952, konkrétne v hre “Nim” a odvtedy sa nejakým štýlom zakomponovala umelá inteligencia do následovných hier kam patria aj súčasne hry ako je Overwatch, Counter-Strike: Global Offensive alebo Fortnite.
V tomto článku sa pozrieme na históriu umelej inteligencie vo videohrách, od videohry Nim až po moderné hry, kde si ukážeme umelú inteligenciu na konkrétnych videohrách ako je Alien Isolation. 
Tiež si vysvetlíme najpoužívanejšie algoritmy na základe ktorých funguje umelá inteligencia, ako je algoritmus hľadania cesty a rozhodovania sa, podobne si predstavíme najdôležitejšie výhody a nevýhody čo sa týka implementovania umelej inteligencie, jej správania sa a výnosov, čo sa týka ťaženia používateľských dát.
\end{abstract}



\section{Úvod}

Motivujte čitateľa a vysvetlite, o čom píšete. Úvod sa väčšinou nedelí na časti.

Uveďte explicitne štruktúru článku. Tu je nejaký príklad.
Základný problém, ktorý bol naznačený v úvode, je podrobnejšie vysvetlený v časti~\ref{nejaka}.
Dôležité súvislosti sú uvedené v častiach~\ref{dolezita} a~\ref{dolezitejsia}.
Záverečné poznámky prináša časť~\ref{zaver}.



\section{História umelej inteligencie vo videohrách} \label{kapitola2}
Pod pojmom umelá inteligencia si predstavme skupinu algoritmov ktorá má zapôsobiť na človeka ako ďalší inteligentný človek.

\subsection{Prvé použitia umelej inteligencie vo videohrách} \label{kapitola2.1}
Umelá inteligencia ako zadefinová v predošlej kapitole bola prvýkrát použitá vo videohre Nim, kde bolo za úlohu poraziť počítač, resp. umelú inteligenciu tým, že na základe herného módu ste nemohli alebo mohli byť posledný na rade, kto odstráňi zápalku alebo zápalky z herného poľa. Pravidlo hry bolo, že ten kto bol na rade, si mohol vybrať, či zoberie z herného poľa len jednu zápalku alebo viac, ale ak sa rozhodol pre viac, tak mohol zápalky brať len z toho radu kde sa nachádza jeho zvoľená prvá zápalka. Kvôli tomuto pravidlu bolo možné vytvoriť algoritmus pre umelú inteligenciu ktorá hrala proti hráčovi a postupne, ako hra prebiehala, umelá inteligencia prepočítavala následujúce možnosti na základe algoritmu a vyberala taký počet zápaliek, čo ju postupne približovalo k výhre.

\subsection{Videohry pred rokom 2000 používajúce umelú inteligenciu} \label{kapitola2.2}
Videohry sa začali vytvárať až po druhej svetovej vojne a počas studenej vojny pretože tieto udalosti pomohli vývoju výpočtovej techniky čo znamenalo, že viacej a viacej ľudí sa podielalo na tomto vývoje nakoľko počítače a výpočtová technika boli doteraz používané prevažne len v druhej svetovej vojne. . V roku 1980 bola vytvorená hra Pac-Man ktorá mala zakomponovanú už o niečo komplikovanejšiu umelú inteligenciu ako videohra Nim, ale stále to boli len jednoduché algoritmy ako sa má umelá inteligencia správať. Vo videohre Pac-Man ale na rozdiel od hry Nim bola umelá inteligencia použitá 4 krát vo forme duchov, ktorý počas hry prenásledujú hráča na základe 4 rozdielnych algoritmov.
Červený duch prenásleduje hráča priamo, nerozmýšla nad tým akými možnými cestami môže hráč utiecť, jednoducho tento duch hľadá najkratšiu cestu k hráčovi a ide po nej.
Ružový duch sa odlišuje od prvého v tom, že nehľadá najkratšiu cestu k hráčovi, ale pred hráča.
Modrý duch má spomedzi ostatných duchov najkomplikovanjší algoritmus a to preto, pretože ako jedinný neberie do úvahy len hráčovu pozíciu, ale aj pozíciu červeného ducha. Ak je hráč ďaleko od červeného ducha tak algoritmus modrého ducha nemá dostatočne presné informácie a často sa modrý duch nachádza ďaleko od hráča. Ako prvé sa zoberie pozícia pred hráčom, rovnako ako pri ružovom duchovi, ďalej sa výtvorí úsečka medzi týmto bodom a červeným duchom, ktorá sa otočí o 180 stupňov a ten bod, ktorý bol na červenom duchovi je teraz cieľ modrého ducha.
Oranžový duch sa správa tak, že ak je viacej ako osem políčok ďaleko od hráča, tak sa správa ako červený duch, ale ak je dostatočne blízko, tak mení svoj algoritmus a ide najkratšou možnou cestou do dolného ľavého rohu. Medzi týmito dvoma algoritmami neustálne prepína.
(Záverny paragraf pre Pac-Man)

\subsection{Umelá inteligencia v moderných videohrách} \label{kapitola2.3}
Neskôr, o 34 rokov vývoja umelej inteligencie vyšla hra s názvom Alien Isolation ktorej cieľom je zistiť, prečo sa vesmírna stanica Sevastopol prestala ozývať. Hráč zistí, že to je kvôli tomu, že na vesmírnej stanici sa nachádzajú votrelci ktorý postupne zabíjajú posádku stanice. Týchto votrelcov riadia dve umelé inteligencie čo už bolo použité aj v hre Left 4 Dead. Tieto dve umelé inteligencie vo videohre Alien Isolation sú rozdelené tak, že jedna neustále vie kde sa hráč nachádza ale tá druhá o tom nevie. Cieľom tej prvej inteligencie je navádzať tú druhú tak, aby vedela na čo sa má sústrediť. Tiež má aj meradlo, čím meria intenzitu stresu ktorú pravdepodobne hráč prežíva, tým, že vidí votrelca alebo je v jeho blízkosti a ak sa toto meradlo naplní, tak prvá umelá inteligencia prikazuje tej druhej aby išla preč od hráča. Ako bolo spomenuté, druhá umelá inteligencia votrelca nikdy nevie presne kde sa hráč nachádza, len na základe poskytnutých informácií prvou umelou inteligenciou postupuje podľa naprogramovaného stromového grafu v ktorom sa nachádza cez 100 možností. Niektoré tieto možnosti sa ale odomknú až po dosiahnutí určitých lokácií kvôli tomu, aby sa votrelec po danom čase hrania nezdal opakujúci.

\begin{figure*}[tbh]
\centering
%\includegraphics[scale=1.0]{diagram.pdf}
\end{figure*}

\section{Algoritmy umelej inteligencie} \label{kapitola3}
Súčasťou umelej inteligencie sú algoritmy ktoré sa snažia napodobniť to, ako sa má daný tvor alebo človek správať. Najčastejšie používané algoritmy v umelej inteligencii vo videohrách je algoritmus hľadania cesty a algoritmus rozhodovania sa.


Môže sa zdať, že problém vlastne nejestvuje\cite{Coplien:MPD}, ale bolo dokázané, že to tak nie je~\cite{Czarnecki:Staged, Czarnecki:Progress}. Napriek tomu, aj dnes na webe narazíme na všelijaké pochybné názory\cite{PLP-Framework}. Dôležité veci možno \emph{zdôrazniť kurzívou}.


\subsection{Algoritmus hladania cesty} \label{kapitola3.1}
Z anglického slova “path finding”, tento algoritmus je jeden z najpoužívanejších v moderných videohrách pretože je veľmi flexibilný, čo vývojári hier veľmi obľubujú. Jeho cieľom je nájsť najkratšiu cestu z bodu A do bodu B. 

\subsection{Algoritmus rozhodovania sa} \label{kapitola3.2}

\paragraph{Veľmi dôležitá poznámka.}
Niekedy je potrebné nadpisom označiť odsek. Text pokračuje hneď za nadpisom.



\section{Dôležitá časť} \label{dolezita}




\section{Ešte dôležitejšia časť} \label{dolezitejsia}




\section{Záver} \label{zaver} % prípadne iný variant názvu



%\acknowledgement{Ak niekomu chcete poďakovať\ldots}


% týmto sa generuje zoznam literatúry z obsahu súboru literatura.bib podľa toho, na čo sa v článku odkazujete
\bibliography{literatura}
\bibliographystyle{plain} % prípadne alpha, abbrv alebo hociktorý iný
\end{document}
