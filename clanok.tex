% Metódy inžinierskej práce

\documentclass[10pt,oneside,slovak,a4paper]{article}

\usepackage[slovak]{babel}
%\usepackage[T1]{fontenc}
\usepackage[IL2]{fontenc} % lepšia sadzba písmena Ľ než v T1
\usepackage[utf8]{inputenc}
\usepackage{graphicx}
\usepackage{url} % príkaz \url na formátovanie URL
\usepackage{hyperref} % odkazy v texte budú aktívne (pri niektorých triedach dokumentov spôsobuje posun textu)

\usepackage{cite}
%\usepackage{times}

\pagestyle{headings}

\title{Umelá inteligencia vo videohrách\thanks{Semestrálny projekt v predmete Metódy inžinierskej práce, ak. rok 2022/23, vedenie: Igor Stupavský}} % meno a priezvisko vyučujúceho na cvičeniach

\author{Dávid Pilný\\[2pt]
	{\small Slovenská technická univerzita v Bratislave}\\
	{\small Fakulta informatiky a informačných technológií}\\
	{\small \texttt{xpilnyd@stuba.sk}}
	}

\date{\small 06.11.2022} % upravte



\begin{document}

\maketitle

\begin{abstract}
V tomto článku si vysvetlíme rozdiely medzi umelou inteligenciou ktorá je používaná v hrách a v priemysle, pozrieme sa na výhody a nevýhody ktoré nastávajú pri použití umelej inteligencie v hrách a tiež si spomenieme dva najpoužívanejšie algoritmy ktoré sa používajú pri tvorbe umelej inteligencie vo videohrách. Taktiež sa pozrieme na evolúciu umelej inteligencie od jej prvého použitia po súčasnosť.
\end{abstract}



\section{Úvod} \label{kapitola1}
<<<<<<< HEAD
Umelá inteligencia je v našich životoch čoraz častejší výskyt a nie je prekvapivé, že ju môžeme vidieť aj vo videohrách, ktoré sa snažia realitu alebo fikciu napodobniť čo najuveriteľnejšie. Vývin umelej inteligencie \ref{kapitola2} už pretrváva vyše polovicu storočia, v podstate od vzniku videohier kde váš súper je niekto iný ako druhý človek \ref{kapitola2.1}. Čím viac sa videohry vývíjajú, tým sa tiež zdokonaľuje umelá inteligencia v imitovaní ľudskej alebo neľudskej bytosti. \ref{kapitola2.3}

Aby sa predišlo zbytočnému kódovaniu algoritmov pre umelú inteligenciu tak sa určité algoritmy, ktoré sa najčastejšie používajú, ako je napríklad algoritmus rozhodovania sa \ref{kapitola3.2} a hľadania cesty. \ref{kapitola3.1}
=======
<<<<<<< HEAD
Umelá inteligencia je v našich životoch čoraz častejší výskyt a nie je prekvapivé, že ju môžeme vidieť aj vo videohrách, ktoré sa snažia realitu alebo fikciu napodobniť čo najuveriteľnejšie. Vývin umelej inteligencie \ref{kapitola2} už pretrváva vyše polovicu storočia, v podstate od vzniku videohier kde váš súper je niekto iný ako druhý človek \ref{kapitola2.1}. Čím viac sa videohry vývíjajú, tým sa tiež zdokonaľuje umelá inteligencia v imitovaní ľudskej alebo neľudskej bytosti. \ref{kapitola2.3}
=======
>>>>>>> 8943ad256222da6cf41d66442d2a2089834d7b5a

Aby sa predišlo zbytočnému kódovaniu algoritmov pre umelú inteligenciu tak sa určité algoritmy, ktoré sa najčastejšie používajú, ako je napríklad algoritmus rozhodovania sa \ref{kapitola3.2} a hľadania cesty. \ref{kapitola3.1}

V tomto článku sa pozrieme na históriu umelej inteligencie, od jej prvého použitia až po súčasnosť \ref{kapitola2}. Následne si predstavíme dva najpoužívanejšie algoritmy ktorými sú algoritmus hľadania cesty a rozhodovania sa \ref {kapitola3}. Ďalej si rozoberieme rozdiely medzi umelou inteligenciou v hrách a mimo hier \ref{kapitola4} a ako posledné sa pozrieme na výhody a nevýhody implementovania umelej inteligencie.
>>>>>>> 7743fd8fce3a737c2f6325419519ce1e1984d949



\section{História umelej inteligencie vo videohrách} \label{kapitola2}
Pod pojmom umelá inteligencia si predstavme skupinu algoritmov a inštrukcií, ktorá má zapôsobiť na človeka ako ďalší inteligentný človek, respektívne iná inteligentná forma života ak sa pokúšame imitovať napríklad nejaké zviera alebo .

\subsection{Prvé použitia umelej inteligencie vo videohrách} \label{kapitola2.1}
Umelá inteligencia ako zadefinová v predošlej kapitole bola prvýkrát použitá vo videohre Nim, kde bolo za úlohu poraziť počítač, resp. umelú inteligenciu tým, že na základe herného módu ste nemohli alebo mohli byť posledný na rade, kto odstráňi zápalku alebo zápalky z herného poľa. Pravidlo hry bolo, že ten kto bol na rade, si mohol vybrať, či zoberie z herného poľa len jednu zápalku alebo viac, ale ak sa rozhodol pre viac, tak mohol zápalky brať len z toho radu kde sa nachádza jeho zvoľená prvá zápalka. Kvôli tomuto pravidlu bolo možné vytvoriť algoritmus pre umelú inteligenciu ktorá hrala proti hráčovi a postupne, ako hra prebiehala, umelá inteligencia prepočítavala následujúce možnosti na základe algoritmu a vyberala taký počet zápaliek, čo ju postupne približovalo k výhre.

\subsection{Videohry pred rokom 2000 používajúce umelú inteligenciu} \label{kapitola2.2}
Videohry sa začali vytvárať až po druhej svetovej vojne a počas studenej vojny pretože tieto udalosti pomohli vývoju výpočtovej techniky čo znamenalo, že viacej a viacej ľudí sa podielalo na tomto vývoje nakoľko počítače a výpočtová technika boli doteraz používané prevažne len v druhej svetovej vojne. V roku 1980 bola vytvorená hra Pac-Man ktorá mala zakomponovanú už o niečo komplikovanejšiu umelú inteligenciu ako videohra Nim, ale stále to boli len jednoduché algoritmy. Vo videohre Pac-Man ale na rozdiel od hry Nim bola umelá inteligencia použitá 4 krát vo forme duchov, ktorý počas hry prenásledujú hráča na základe 4 rozdielnych algoritmov \cite{PacmanAI}.

Červený duch prenásleduje hráča priamo, nerozmýšla nad tým akými možnými cestami môže hráč utiecť, jednoducho tento duch hľadá najkratšiu cestu k hráčovi a ide po nej.

Ružový duch sa odlišuje od prvého v tom, že nehľadá najkratšiu cestu k hráčovi, ale pred hráča.

Modrý duch má spomedzi ostatných duchov najkomplikovanjší algoritmus a to preto, pretože ako jedinný neberie do úvahy len hráčovu pozíciu, ale aj pozíciu červeného ducha. Ak je hráč ďaleko od červeného ducha tak algoritmus modrého ducha nemá dostatočne presné informácie a často sa modrý duch nachádza ďaleko od hráča. Ako prvé sa zoberie pozícia pred hráčom, rovnako ako pri ružovom duchovi, ďalej sa výtvorí úsečka medzi týmto bodom a červeným duchom, ktorá sa otočí o 180 stupňov a ten bod, ktorý bol na červenom duchovi je teraz cieľ modrého ducha.

Oranžový duch sa správa tak, že ak je viacej ako osem políčok ďaleko od hráča, tak sa správa ako červený duch, ale ak je dostatočne blízko, tak mení svoj algoritmus a ide najkratšou možnou cestou do dolného ľavého rohu. Medzi týmito dvoma algoritmami neustálne prepína.
(Záverny paragraf pre Pac-Man)

\subsection{Umelá inteligencia v moderných videohrách} \label{kapitola2.3}
<<<<<<< HEAD
Neskôr, o 34 rokov vývoja umelej inteligencie vyšla hra s názvom Alien Isolation ktorej cieľom je zistiť, prečo sa vesmírna stanica Sevastopol prestala ozývať. Hráč zistí, že to je kvôli tomu, že na vesmírnej stanici sa nachádzajú votrelci ktorý postupne zabíjajú posádku stanice. Týchto votrelcov riadia dve umelé inteligencie \cite{AlienIsolationAI}. Sú rozdelené tak, že jedna neustále vie kde sa hráč nachádza ale tá druhá o tom nevie. Cieľom tej prvej inteligencie je navádzať tú druhú tak, aby vedela na čo sa má sústrediť. Tiež má aj meradlo, čím meria intenzitu stresu ktorú pravdepodobne hráč prežíva, tým, že vidí votrelca alebo je v jeho blízkosti a ak sa toto meradlo naplní, tak prvá umelá inteligencia prikazuje tej druhej aby išla preč od hráča. Ako bolo spomenuté, druhá umelá inteligencia votrelca nikdy nevie presne kde sa hráč nachádza, len na základe poskytnutých informácií prvou umelou inteligenciou postupuje podľa naprogramovaného stromového grafu v ktorom sa nachádza cez 100 možností. Niektoré tieto možnosti sa ale odomknú až po dosiahnutí určitých lokácií kvôli tomu, aby sa votrelec po danom čase hrania nezdal opakujúci.
=======
<<<<<<< HEAD
Neskôr, o 34 rokov vývoja umelej inteligencie vyšla hra s názvom Alien Isolation ktorej cieľom je zistiť, prečo sa vesmírna stanica Sevastopol prestala ozývať. Hráč zistí, že to je kvôli tomu, že na vesmírnej stanici sa nachádzajú votrelci ktorý postupne zabíjajú posádku stanice. Týchto votrelcov riadia dve umelé inteligencie \cite{AlienIsolationAI}. Sú rozdelené tak, že jedna neustále vie kde sa hráč nachádza ale tá druhá o tom nevie. Cieľom tej prvej inteligencie je navádzať tú druhú tak, aby vedela na čo sa má sústrediť. Tiež má aj meradlo, čím meria intenzitu stresu ktorú pravdepodobne hráč prežíva, tým, že vidí votrelca alebo je v jeho blízkosti a ak sa toto meradlo naplní, tak prvá umelá inteligencia prikazuje tej druhej aby išla preč od hráča. Ako bolo spomenuté, druhá umelá inteligencia votrelca nikdy nevie presne kde sa hráč nachádza, len na základe poskytnutých informácií prvou umelou inteligenciou postupuje podľa naprogramovaného stromového grafu v ktorom sa nachádza cez 100 možností. Niektoré tieto možnosti sa ale odomknú až po dosiahnutí určitých lokácií kvôli tomu, aby sa votrelec po danom čase hrania nezdal opakujúci.
=======
<<<<<<< HEAD
Neskôr, o 34 rokov vývoja umelej inteligencie vyšla hra s názvom Alien Isolation ktorej cieľom je zistiť, prečo sa vesmírna stanica Sevastopol prestala ozývať. Hráč zistí, že to je kvôli tomu, že na vesmírnej stanici sa nachádzajú votrelci ktorý postupne zabíjajú posádku stanice. Týchto votrelcov riadia dve umelé inteligencie \cite{AlienIsolationAI}. Sú rozdelené tak, že jedna neustále vie kde sa hráč nachádza ale tá druhá o tom nevie. Cieľom tej prvej inteligencie je navádzať tú druhú tak, aby vedela na čo sa má sústrediť. Tiež má aj meradlo, čím meria intenzitu stresu ktorú pravdepodobne hráč prežíva, tým, že vidí votrelca alebo je v jeho blízkosti a ak sa toto meradlo naplní, tak prvá umelá inteligencia prikazuje tej druhej aby išla preč od hráča. Ako bolo spomenuté, druhá umelá inteligencia votrelca nikdy nevie presne kde sa hráč nachádza, len na základe poskytnutých informácií prvou umelou inteligenciou postupuje podľa naprogramovaného stromového grafu v ktorom sa nachádza cez 100 možností. Niektoré tieto možnosti sa ale odomknú až po dosiahnutí určitých lokácií kvôli tomu, aby sa votrelec po danom čase hrania nezdal opakujúci.
=======
Neskôr, o 34 rokov vývoja umelej inteligencie vyšla hra s názvom Alien Isolation ktorej cieľom je zistiť, prečo sa vesmírna stanica Sevastopol prestala ozývať. Hráč zistí, že to je kvôli tomu, že na vesmírnej stanici sa nachádzajú votrelci ktorý postupne zabíjajú posádku stanice. Týchto votrelcov riadia dve umelé inteligencie. Sú rozdelené tak, že jedna neustále vie kde sa hráč nachádza ale tá druhá o tom nevie. Cieľom tej prvej inteligencie je navádzať tú druhú tak, aby vedela na čo sa má sústrediť. Tiež má aj meradlo, čím meria intenzitu stresu ktorú pravdepodobne hráč prežíva, tým, že vidí votrelca alebo je v jeho blízkosti a ak sa toto meradlo naplní, tak prvá umelá inteligencia prikazuje tej druhej aby išla preč od hráča. Ako bolo spomenuté, druhá umelá inteligencia votrelca nikdy nevie presne kde sa hráč nachádza, len na základe poskytnutých informácií prvou umelou inteligenciou postupuje podľa naprogramovaného stromového grafu v ktorom sa nachádza cez 100 možností. Niektoré tieto možnosti sa ale odomknú až po dosiahnutí určitých lokácií kvôli tomu, aby sa votrelec po danom čase hrania nezdal opakujúci.
>>>>>>> df1749b91d3f0cfff9bef32fe2c36c73d5cd7ac0
>>>>>>> 8943ad256222da6cf41d66442d2a2089834d7b5a
>>>>>>> 7743fd8fce3a737c2f6325419519ce1e1984d949


\section{Algoritmy umelej inteligencie} \label{kapitola3}
Súčasťou umelej inteligencie sú algoritmy ktoré sa snažia napodobniť to, ako sa má daný tvor alebo človek správať. Najčastejšie používané algoritmy v umelej inteligencii vo videohrách je algoritmus hľadania cesty a algoritmus rozhodovania sa.

\subsection{Algoritmus hladania cesty} \label{kapitola3.1}
Z anglického slova “path finding”, tento algoritmus je jeden z najpoužívanejších v moderných videohrách pretože je veľmi flexibilný, čo vývojári hier veľmi obľubujú. Jeho cieľom je nájsť najkratšiu cestu z bodu A do bodu B. 

\subsection{Algoritmus rozhodovania sa} \label{kapitola3.2}
Ako z názvu vyplýva, tento algoritmus určuje čo a na základe akých okolností vykoná umelá inteligencia. Vstupom sú dáta z okolia kde sa umelá inteligencia nachádza a z týchto vstupov ktorých môže byť aj veľmi veľa usudzuje, čo má spraviť ďalej. Ako taký príklad môže byť, že hráč strieľa po umelej inteligencii ktorá na základe tohto vstupu, sa rozhodne, že sa skryje za najbližšiu prekážku a začne opätovať streľbu. Takéto rozhodnutie si môžeme všimnúť v mnohých moderných hrách.


\section{Typy umelej inteligencie} \label{kapitola4}
Umelá inteligencia v hrách a v iných odvetiach sa líši v tom, že tá v hrách je vytvorená väčšinou tak, aby napodobňovala inú inteligentnú formu života. Kvôli tomuto nikdy nedosiahne svôj úplný potenciál a je nútená v mnohých prípadoch byť porazená hráčom. Napríklad spomenutá hra Pac-Man, každý duch má špeciálne vytvorenú umelú inteligenciu tak, aby hráč mal šancu vyhrať danú úroveň a kvôli tomuto je táto hra zábavná. Umelá inteligencia mimo hier, ktorými sú napríklad Alexa alebo Google Assistant, používajú strojové učenie (Machine Learning) čo je vlastne podkategóriou umelej inteligencie a pomáha jej zlepšovať sa v ich práci bez vypisovania veľkého počtu riadkov kódu. Jednoducho, umelá inteligencia v hrách sa najčastejšie riadi podľa presne vytýčených algoritmov na rozdiel od tej využívanej mimo hier, ktorá tiež používa nejaké algoritmy ale hlavným rozdielom je použitie strojového učenia a umožnenie dovŕšenia úplného potenciálu, nakoľko nie je limitovaná imitáciou nejakej inteligentnej formy života.


\section{Výhody a nevýhody umelej inteligencie vo videohrách} \label{kapitola5}


\section{Záver} \label{zaver} % prípadne iný variant názvu



%\acknowledgement{Ak niekomu chcete poďakovať\ldots}


% týmto sa generuje zoznam literatúry z obsahu súboru literatura.bib podľa toho, na čo sa v článku odkazujete
\bibliography{literatura}
\bibliographystyle{plain} % prípadne alpha, abbrv alebo hociktorý iný
\end{document}
